%
% Complete documentation on the extended LaTeX markup used for Insight
% documentation is available in ``Documenting Insight'', which is part
% of the standard documentation for Insight.  It may be found online
% at:
%
%     http://www.itk.org/

\documentclass{InsightArticle}

\usepackage[dvips]{graphicx}

%%%%%%%%%%%%%%%%%%%%%%%%%%%%%%%%%%%%%%%%%%%%%%%%%%%%%%%%%%%%%%%%%%
%
%  hyperref should be the last package to be loaded.
%
%%%%%%%%%%%%%%%%%%%%%%%%%%%%%%%%%%%%%%%%%%%%%%%%%%%%%%%%%%%%%%%%%%
\usepackage[dvips,
bookmarks,
bookmarksopen,
backref,
colorlinks,linkcolor={blue},citecolor={blue},urlcolor={blue},
]{hyperref}

\usepackage{bm}
\usepackage{amsmath}
\usepackage{amsfonts,amssymb}

\usepackage{subfigure}


%  This is a template for Papers to the Insight Journal.
%  It is comparable to a technical report format.

% The title should be descriptive enough for people to be able to find
% the relevant document.
\title{A Simple Command Line Argument Parser}

%
% NOTE: This is the last number of the "handle" URL that
% The Insight Journal assigns to your paper as part of the
% submission process. Please replace the number "1338" with
% the actual handle number that you get assigned.
%
\newcommand{\IJhandlerIDnumber}{3258}

% Increment the release number whenever significant changes are made.
% The author and/or editor can define 'significant' however they like.
\release{1.0}

% At minimum, give your name and an email address.  You can include a
% snail-mail address if you like.
\author{Marius Staring$^{1}$ and Stefan Klein$^{2}$ and David Doria$^{3}$ }
\authoraddress{$^{1}$Division of Image Processing, Leiden University Medical Center, Leiden, The Netherlands\\
               $^{2}$Biomedical Imaging Group Rotterdam, Departments of Radiology \& Medical Informatics, Erasmus MC, Rotterdam, The Netherlands\\
	       $^{3}$Rensselaer Polytechnic Institue}

\begin{document}

%
% Add hyperlink to the web location and license of the paper.
% The argument of this command is the handler identifier given
% by the Insight Journal to this paper.
%
\IJhandlefooter{\IJhandlerIDnumber}


\ifpdf
\else
   %
   % Commands for including Graphics when using latex
   %
   \DeclareGraphicsExtensions{.eps,.jpg,.gif,.tiff,.bmp,.png}
   \DeclareGraphicsRule{.jpg}{eps}{.jpg.bb}{`convert #1 eps:-}
   \DeclareGraphicsRule{.gif}{eps}{.gif.bb}{`convert #1 eps:-}
   \DeclareGraphicsRule{.tiff}{eps}{.tiff.bb}{`convert #1 eps:-}
   \DeclareGraphicsRule{.bmp}{eps}{.bmp.bb}{`convert #1 eps:-}
   \DeclareGraphicsRule{.png}{eps}{.png.bb}{`convert #1 eps:-}
\fi

\maketitle

\ifhtml
\chapter*{Front Matter\label{front}}
\fi

\begin{abstract}
\noindent This document describes the implementation of a simple
command line argument parser using the Insight Toolkit ITK
\url{www.itk.org}. Such a parser may be useful for use in the
examples of the ITK.

This paper is accompanied with the source code.
\end{abstract}

\IJhandlenote{\IJhandlerIDnumber}

\tableofcontents

%%%%%%%%%%%%%%%%%%%%%%%%%%%%%%%%%%%%%%%%%%%%%%%%%%%%%%%%%%%%%%%

\section{Introduction}

Command line argument parsing is a common task for many (small)
programs. There are many tools already available for parsing command
line options. These tools, however, typically require the inclusion
of a large library (Boost, Wx, etc). The advantage of our proposed
parser is that it is very small and non-intrusive while remaining
versatile enough to fulfill the needs of the ITK examples and tests. The
proposed parser has been used extensively for many years in the
toolkit \texttt{praxix}: \url{http://code.google.com/p/praxix/}.

\section{Argument Syntax}
The tool assumes that arguments are passed in key-value
combinations. For example:
\begin{verbatim}
 -key value1 $\cdots$ valueN
\end{verbatim}
Keys are identified as a ``-'' followed by a string. Subsequent entries that are not keys are the values. Zero or more values can be specified for each key.

%%%%%%%%%%%%%%%%%%%%%%%%%%%%%%
\section{Usage}

\subsection{Instantiating the Parser}
The parser is created by instantiating it as you would a normal ITK class, and then passing the \verb|argc| and \verb|argv| from \verb|main| to the \verb|SetCommandLineArguments| function:
\small
\begin{verbatim}
itk::CommandLineArgumentParser::Pointer parser = itk::CommandLineArgumentParser::New();
parser->SetCommandLineArguments( argc, argv );
\end{verbatim}
\normalsize 

\subsection{Checking If An Argument Exists}
The developer can check to see if an argument has been passed without retrieving its value with:
\small
\begin{verbatim}
bool exists = parser->ArgumentExists( "-key" );
\end{verbatim}
\normalsize 

\subsection{Retrieving a Single Argument}
To retrieve a single argument from the command line, one should use the \verb|GetCommandLineArgument| function. The argument can be of any type. For example, to fetch a string argument, one can use:

\small
\begin{verbatim}
std::string stringarg;
parser->GetCommandLineArgument( "-mystring", stringarg );
\end{verbatim}
\normalsize 

The \verb|GetCommandLineArgument| function returns a bool indicating whether or not the argument existed:

\small
\begin{verbatim}
std::string stringarg;
bool exists = parser->GetCommandLineArgument( "-mystring", stringarg );
\end{verbatim}

This functionality allows an explicit call to \verb|ArgumentExists| to be avoided if the developer would like to retrieve and check for an argument simultaneously.


\subsection{Retrieving an Argument List}
It is also possible to retrieve a list of arguments from the command line. The argument must all be of the same type. This is done simply by retrieving a \verb|vector| of the argument type. For example, to fetch a list of filenames (strings), one can use:

\small
\begin{verbatim}
std::vector<std::string> filenames;
parser->GetCommandLineArgument( "-filenames", filenames);
\end{verbatim}
\normalsize 

\subsection{Marking an Argument as Required}
In many cases there is no reasonable default for an argument. For example, often the input file should be specified with \verb|-in|. If this is not specified, the program would not know which image to operate on. To simplify the handling of this case, we provide:

\small
\begin{verbatim}
parser->MarkArgumentAsRequired( "-out", "The output filename." );
\end{verbatim}
\normalsize 

\subsection{Requiring Exactly One of a Group of Arguments}
Often a program requires ``exactly one of'' a group of arguments to be specified. This is possible via the \verb|MarkExactlyOneOfArgumentsAsRequired| function:

\small
\begin{verbatim}
std::vector<std::string> exactlyOneArguments;
exactlyOneArguments.push_back("-sz");
exactlyOneArguments.push_back("-in");

parser->MarkExactlyOneOfArgumentsAsRequired(exactlyOneArguments);
\end{verbatim}
\normalsize 

This will cause the \verb|CheckForRequiredArguments| (see Section \ref{subsec:ValidatingInputArguments}) test to fail in the case where neither \verb|-sz| nor \verb|-in| is specified, or when both have been specified. The test will pass if exactly one of the arguments in the list has been set. There is no limit to the number of such lists that can be specified.

\subsection{Setting and Showing Help Text}
The user may want to inquire as to what the program can do. The developer can specify HelpText which can be shown in several ways. The text is specified with the \verb|SetProgramHelpText| function:

\small
\begin{verbatim}
parser->SetProgramHelpText("Some help text.");
\end{verbatim}
\normalsize 

The text will be shown if the user passes ``--help'', ``-help'', or ``-h''. The text is also shown if the input arguments are not valid (see Section \ref{subsec:ValidatingInputArguments}).

\subsection{Validating Input Arguments}
\label{subsec:ValidatingInputArguments}
Once all required arguments and groups of arguments are set, the \verb|CheckForRequiredArguments| function should be called to ensure everything that has been requested has also been specified. If all required conditions are not met, the HelpText is output automatically. The developer can also choose to do something in addition, but the usual usage is simply to quit.

\small
\begin{verbatim}
bool validateArguments = parser->CheckForRequiredArguments();

if(!validateArguments)
{
  return EXIT_FAILURE;
}
\end{verbatim}
\normalsize 


\subsection{Default values}
\subsection{Single Argument}
Arguments can be initialized to default values, which will be left
untouched if the key is not provided at the command line. 

In this example:
\small
\begin{verbatim}
int test = 2;
parser->GetCommandLineArgument( "-test", test );
\end{verbatim}
\normalsize 

if \verb|-test| was not specified on the command line, the value of \verb|test| will remain \verb|2| after the call to \verb|GetCommandLineArgument|. If \verb|-test| was specified, then \verb|test| will take the value of the \verb|value| associated with the \verb|-test| key.

\subsection{List of Arguments}
If an argument is initialized with a vector of $\texttt{size} > 1$, and if
only one (1) argument is provided in the command line, a vector of
size $\texttt{size}$ is created and filled with the single argument.

%%%%%%%%%%%%%%%%%%%%%%%%%%%%%%%%%%%%%%%%%%%%%%%%%%%%%%%%%%%%%%%

\section{Internals}

Internally, the command line arguments are stored in an
\code{std::map} of the argument or key as an \code{std::string}
together with the index. We make use of the casting functionality of
string streams to automatically cast the stored string to the
requested type.

%%%%%%%%%%%%%%%%%%%%%%%%%%%%%%%%%%%%%%%%%%%%%%%%%%%%%%%%%%%%%%%

\section{Examples of usage}

In this section we demonstrate the ease with which the parser can be used for many different types of arguments.


\subsection{Passing a boolean argument}

A boolean argument is set to true by simply passing the flag:

\small\code{executablename -z}\normalsize

This code \small
\begin{verbatim}
bool compress = parser->ArgumentExists( "-z" );
\end{verbatim}
\normalsize will return true if \texttt{-z} was part of the list of
arguments, and false otherwise.

\subsection{Passing an integer argument}

\small\code{executablename -num 5}\normalsize

\small
\begin{verbatim}
int intValue; // no default
bool retnum = parser->GetCommandLineArgument( "-num", intValue );
\end{verbatim}
\normalsize

\subsection{Passing a floating argument}

\small\code{executablename -pi 3.1415926}\normalsize

\small
\begin{verbatim}
float pi = 3.0; // default
bool retpi = parser->GetCommandLineArgument( "-pi", pi );
\end{verbatim}
\normalsize

\subsection{Passing a string argument}

\small\code{executablename -m Wavelet}\normalsize

\small
\begin{verbatim}
std::string method = "Fourier"; // default
bool retm = parser->GetCommandLineArgument( "-m", method );
\end{verbatim}
\normalsize

\subsection{Passing a vector argument}

\small\code{executablename -in file1.png file2.bmp}\normalsize

\small
\begin{verbatim}
std::vector<std::string> inputFileNames; // no default
bool retin = parser->GetCommandLineArgument( "-in", inputFileNames );
\end{verbatim}
\normalsize

\subsection{Setting argument defaults}

Behavior when the argument is passed:

\small\code{executablename -pA 2 5 9 4}\normalsize

\small
\begin{verbatim}
std::vector<int> vecA( 3, 1 ); // using default values
bool retpA = parser->GetCommandLineArgument( "-pA", vecA );
// The value of vecA is {2, 5, 9, 4}
\end{verbatim}
\normalsize

Behavior when the argument is NOT passed:

\small\code{executablename}\normalsize

\small
\begin{verbatim}
std::vector<int> vecA( 3, 1 ); // using default values
bool retpA = parser->GetCommandLineArgument( "-pA", vecA );
// The value of vecA is {1, 1, 1}
\end{verbatim}
\normalsize

\subsection{Combined example}

Multiple arguments can be passed to the command as follows:

\small\code{executablename -in input1.mhd input2.mhd -z -out output.png -pi 3.1415926535}\normalsize

These arguments are retrieved identically as in the individual examples above.

%%%%%%%%%%%%%%%%%%%%%%%%%%%%%%%%%%%%%%%%%%%%%%%%%%%%%%%%%%%%%%%

\section{Conclusion}

This document describes the implementation of a simple command line
argument parser using the Insight Toolkit ITK \url{www.itk.org}.
Such a parser may be useful for use in the examples of the ITK or on
it's wiki.

%%%%%%%%%%%%%%%%%%%%%%%%%%%%%%%%%%%%%%%%%%%%%%%%%%%%%%%%%%%%%%%

%\appendix
%\section{This is an Appendix}

% Bibliography
%\bibliographystyle{plain}
%\bibliography{InsightJournal}

\end{document}
